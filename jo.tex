\section{Join Reordering without statistics}
In prevent section we looked into current state of affair for Join Reordering and why would it not work for Big Data systems. We will like to propose a novel approach in this section to reorder Join just on the basis of table sizes without any elaborate Table or Column Statistics as required by existing approaches.
We have implemented this approach in Apache Spark and will show how our approach gives upto X improvement in TPCDS queries in our empirical evaluation.

\subsection{Heuristics}

Let us look at the assumptions and heuristics used for the approach.

\subsubsection{Assumptions}:

\begin{itemize}
\item In Star Schema Join, the Dimension table’s primary key will be joined with the Fact table’s foreign key.
\item Cardinality of Fact Table joined with Filtered Dimension table will be less than that of the Fact table.
\item So based on last assumption, it follows that if Fact Join with Filtered Dimension happens earlier in multi-way join, it can benefit later joins.
\end{itemize}

To illustrate assumption above, let's look at this query:
\texttt{select * from fact\_table f join dim1 d1 on f.k1 = d1.pk join dim2 d2 on f.k2 = d2.pk where d2.date = ‘06-10-2019’}. If there is a filter on \texttt{dim2} which is smaller than \texttt{dim1} assumption is it’s join with \texttt{fact\_table} would be smaller than \texttt{fact\_table}. Hence, joining \texttt{fact\_table} with \texttt{dim2} and then with \texttt{dim1} will be beneficial.

\subsubsection{Heuristics for detecting Star schema}
In the lack of statistics, we have to use only table sizes and query structure to determine if a Join is a fact table. We are using following heuristics to determine them:

\begin{itemize}
\item Table involved in all of the joins should be a fact table.
\item At least half of the inner joins in the query should involve the fact table.
\item The ratio of the size of the largest dimensions table to fact table size has to be lesser than a certain threshold.
\item None of the dimensions tables is partitioned.
\end{itemize}








