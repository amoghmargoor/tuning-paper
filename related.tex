\section{Related Work}
\label{sec:relatedwork}
Most relational databases have auto-tuning tools for physical database design. 
Self-tuning Database Systems: A Decade of Progress~\cite{Chaudhuri}
 has a good survey of research and tools in this area. Most database systems have 
not put similar attention to determine best values of configuration parameters 
based on workloads. Traditionally effort has focused on specific class of parameters 
(e.g.\cite{Storm} ) or on ranking
critical configuration parameters\cite{DebnathLM08}. All these techniques
are based on using the optimizer cost model to simulate database behavior. The disadvantage is
that the cost model may not model the effects of all important parameters.

The popularity of Apache Hadoop ecosystem (Apache Hive, Apache Spark etc) increased interest
in choosing configuration parameters automatically. 
\eat{
	The developers of these systems exposed
engines parameters, documented them and encouraged users to change them based on their workload.
The main motivation was that the type of workloads and data processed on these systems were varied
and the default values were suboptimal. It was assumed that the administrators have to manipulate
parameters for a successful installation. Similar to tools in database systems, there are two possible approaches - model based and execution based.
}
Starfish \cite{herodotou2011starfish} was one of the first to model hadoop performance. Starfish 
consists of a \textit{profiler}, \textit{what-if} engine and a \textit{cost-based model}\cite{herodotou2011profiling}. 
\eat{
The \textit{profiler} collects statistics from previous runs of a customer workload 
such as bytes flowing through the system and time taken using dynamic instrumentation. The profiler provides a view of timings (wall clock timings of various phases),
data flow (bytes input/output in every phase) and resources (CPU, networking, memory) used. The \textit{what-if engine} simulates 
the behavior of the Map Reduce engine and can predict the effect of changing the value 
of a configuration parameter.
The \textit{profiler} collects statistics from previous runs of a customer workload.
 The \textit{cost based optimizer} uses the \textit{what-if engine}
and recursive random search (RSS) for tuning the parameters for a Hadoop job.} 
Starfish was extended to automatically size clusters on cloud platforms using \textit{Elastisizer}\cite{HerodotouNoOne}.
Compared to Starfish which is applicable to all types of workloads on a MapReduce Engine, our approach differs in focussing on SQL workloads only. The focus on 
SQL workloads reduces the number of metrics required in profiler phase as well as the number of configuration parameters in the what-if engine.
This approach has very practical benefits. An approach like Starfish has to consider every engine like Apache Hive, Apache Spark, Apache Impala and PrestoDb separately
as the internal architectures are very different. By simplifying the model to a few common characteristics of all SQL engines, our approach can be extended
to other SQL engines much more easily. For example, in this paper we provide models for both Apache Hive and Apache Spark. Moreover these projects move 
fast and the internal architectures change often in every release. The chances of requiring changes is lesser if the model is based on few parameters important for SQL workloads.
The disadvantage of our approach is that it is not applicable to all types of workloads. 


There are multiple approaches \cite{wu2013self} \cite{lama2012aroma} that use machine learning techniques
to cluster jobs based on their profiles and choose the most optimal cluster configuration. \eat{Aroma \cite{lama2012aroma} optimizes resource allocation and cost of jobs. \eat{In
the offline phase, using a training set, the jobs are clustered
(using variants of k-means) according to their respective signatures.
In the online phase, Aroma trains a SVM which makes
accurate and fast prediction of a job's performance for various
configuration parameters and input data sizes.} The optimization function in Aroma considers
the machine type, number of machines and unit cost of the machine types with a constraint on
run time of the query. Our approach uses a rule based mathematical model instead of statistical or machine learning approaches. }

An alternative school of thought is the iterative method, in which experiments
with alternate values of configuration parameters are executed instead of being simulated in a statistical or mathematical model. They don't need to rely on accurate models for complex data engines which are hard to build. The obvious disadvantage is that experiments cost money and the main 
goal of these approaches is to make every experiment count.
iTuned\cite{Duan} is one of the first attempts in this direction to holistically 
tune configuration parameters in modern database systems. \eat{It consists of a planner
which plans experiments and an executor that run experiments to choose the best parameters
and the best values for a specific workload. iTuned is also a good example of a system that
	ran experiments on user workloads and queries compared to using models to simulate data engines.}
CherryPick\cite{LiMinZeng} and Sandeep Kumar et al.\cite{KumarPLPGB16} apply this approach to Hadoop based systems.\eat{SPSA is used in \cite{KumarPLPGB16}, whereas Bayesian Optimization is used in \cite{LiMinZeng} 
to choose an optimal set of experiments to find the best values for a pre-determined set of configuration parameters.}
We initially tried an iterative execution approach and found it to be impractical in terms of dollar cost\cite{fullpaper}. At the same time it is hard to manage detailed models of complex engine. Thus, 
our preferred approach focuses on SQL-on-Hadoop engines and uses models to eliminate cost of experiments.


 \eat{
CherryPick\cite{LiMinZeng} and Sandeep Kumar et al.\cite{KumarPLPGB16} represent an alternate school of thought.
Mathematical or statistical models for complex data engines is hard to build. Moreover maintenance of these models 
is also hard as these data engines evolve. In this approach, experiments
with alternate values of configuration parameters are executed instead of simulated in a 
statistical or mathematical model. The obvious disadvantage is that experiments cost money and the main 
goal of these approaches is to make every experiment count.
In Section~\ref{section:iter} we tried an iterative execution approach and found it to be impractical in terms of dollar cost. 
At the same time it is hard to manage detailed models of complex engine and therefore
we propose a simple model focused on important SQL operations.
}  
