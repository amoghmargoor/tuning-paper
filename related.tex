\section{Related Work}
\label{sec:relatedwork}
Most relational databases have auto-tuning tools for physical database design. 
Self-tuning Database Systems: A Decade of Progress\cite{Chaudhuri:2007:SDS:1325851.1325856}
 has a good survey of research and tools in this area. Most database systems have 
not put similar attention to determine best values of configuration parameters 
based on workloads. Traditionally effort has focused on specific class of parameters 
(e.g.\cite{Storm:2006:ASM:1182635.1164220} ) or on ranking
critical configuration parameters\cite{DBLP:conf/icde/DebnathLM08}. All these techniques
are based on using the optimizer cost model to simulate database behavior. The disadvantage is
that the cost model may not model the effects of all important parameters.

Ituned\cite{Duan:2009:TDC:1687627.1687767} is one of the first attempts to holistically 
tune configuration parameters in modern database systems. iTuned consists of a planner
which plans experiments and an executor that run experiments to choose the best parameters
and the best values for a specific workload. iTuned is also a good example of a system that
ran experiments on user workloads and queries compared to using models to simulate data engines.
Our preferred approach focuses on SQL-on-Hadoop engine and uses models to eliminate cost of experiments.
 
The popularity of Apache Hadoop ecosystem (Apache Hive, Apache Spark etc) increased interest
in choosing configuration parameters automatically. The developers of these systems exposed
engines parameters, documented them and encouraged users to change them based on their workload.
The main motivation was that the type of workloads and data processed on these systems were varied
and the default values were suboptimal. It was assumed that the administrators have to manipulate
parameters for a successfuly installation. Similar to tools in database systems, there are two schools
of techniques - model based and execution based.

Starfish\cite{herodotou2011starfish} was one of the first to model hadoop performance. Starfish 
consists of a \textit{profiler}, \textit{what-if} engine and a \textit{cost-based model}\cite{herodotou2011profiling}. 
The \textit{profiler} collects statistics from previous runs of a customer workload 
such as bytes flowing through the system and time taken. The \textit{what-if engine} simulates 
the behavior of the Map Reduce engine and can predict the effect of changing the value 
of a configuration parameter. The \textit{cost based optimizer} uses the \textit{what-if engine}
and recursive random search (RSS) for tuning the parameters for a Hadoop job. The components of
Starfish were extended with a new component - \textit{Elastisizer}\cite{Herodotou:2011:NOS:2038916.2038934}
- to automatically size clusters on cloud platforms. Our approach uses a very simple model. This approach
is practical because we focus on common SQL operations. 

There are multiple approaches \cite{wu2013self} \cite{lama2012aroma} that use machine learning techniques
to cluster jobs based on their profiles. Based on the profile, the most optimal cluster configuration is
used. Aroma \cite{lama2012aroma} optimizes resource allocation and cost of jobs. In
the offline phase, using a training set, the jobs are clustered
(using variants of k-means) according to their respective signatures.
In the online phase, Aroma trains a SVM which makes
accurate and fast prediction of a job's performance for various
configuration parameters and input data sizes. The optimization function in Aroma considers
the machine type, number of machines and unit cost of the machine types with a constraint on
run time of the query. Our approach uses a mathematical model instead of statistical or machine learning approaches. 
 
CherryPick\cite{Li:2014:MMO:2600212.2600229} and Sandeep Kumar et al.\cite{KumarPLPGB16} represent an alternate school of thought.
Mathematical or statistical models for complex data engines is hard to build. Moreover maintenance of these models 
is also hard as these data engines evolve. In this approach, experiments
with alternate values of configuration parameters are are executed instead of simulated in a 
statistical or mathematical model. The obvious disadvantage is that experiments cost money and the main 
goal of these approaches is to make every experiment count. In \cite{KumarPLPGB16}, the authors use SPSA and Bayesian Optimization in \cite{Li:2014:MMO:2600212.2600229} 
to choose a optimal set of experiments to find the best values for a pre-determined set of configuration parameters.
In Section~\ref{section:iter} we tried an iterative execution approach and found it to be impractical in terms of dollar cost. 
At the same time it is hard to manage detailed models of complex engine and therefore
we propose a simple model specialized for SQL-on-Hadoop engines.
  
