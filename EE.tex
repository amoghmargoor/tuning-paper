\documentclass[sigconf]{acmart}

\usepackage{listings}
\usepackage{booktabs} % For formal tables


% Copyright
%\setcopyright{none}
%\setcopyright{acmcopyright}
%\setcopyright{acmlicensed}
\setcopyright{rightsretained}
%\setcopyright{usgov}
%\setcopyright{usgovmixed}
%\setcopyright{cagov}
%\setcopyright{cagovmixed}


% DOI
\acmDOI{10.475/123_4}

% ISBN
\acmISBN{123-4567-24-567/08/06}

%Conference
\acmConference[WOODSTOCK'97]{ACM Woodstock conference}{July 1997}{El
  Paso, Texas USA} 
\acmYear{1997}
\copyrightyear{2016}

\acmPrice{15.00}


\begin{document}
\title{Optimizing SQL Workloads for Big Data}

\author{Prasad Deshpande}
\affiliation{%
  \institution{Qubole India}
  \city{Bengaluru}
  \state{Karnataka}
  \country{India}
}
\email{pmd@qubole.com}

\author{Amogh Margoor}
\affiliation{%
  \institution{Qubole India}
  \city{Bengaluru}
  \state{Karnataka}
  \country{India}
}
\email{amoghm@qubole.com}

\author{Rajat Venkatesh}
\affiliation{%
  \institution{Qubole India}
  \city{Bengaluru}
  \state{Karnataka}
  \country{India}
}
\email{rvenkatesh@qubole.com}

% The default list of authors is too long for headers}
\renewcommand{\shortauthors}{P. Deshpande et al.}


\begin{abstract}
Data explosion has pushed various organisation and enterprises towards large scale Big Data adoption. Due to popularity of SQL as a data language to query, manipulate and visualize data there are lot technologies that offer SQL-on-Big Data like Apache Hive, SparkSQL, Presto. These SQL-on-Big Data technologies have seen wide adoption. However, auto tuning them for performance or cost is still a big challenge and is mostly done manually by experts around. In this paper, we would evaluate and propose novel methods to optimize SQL workloads on Big Data. These query engines offer plethora of configurations to tune. General approach is to use rule of thumb to arrive at this, but that may be far from optimal configuration for the job. Our method would come up with mathematical model for Engines that can be used to arrive at optimal configuration. Our method would also suggest instance type of the cluster that would suite a particular workload. As our approach focuses on SQL workloads instead of generic Big Data Workload it has helped in greatly simplifying the model.
\end{abstract}

\keywords{Big Data, SQL, Hadoop performance tuning \LaTeX, text tagging}


\maketitle

\section{Conclusions}
This paragraph will end the body of this sample document.
Remember that you might still have Acknowledgments or
Appendices; brief samples of these
follow.  There is still the Bibliography to deal with; and
we will make a disclaimer about that here: with the exception
of the reference to the \LaTeX\ book, the citations in
this paper are to articles which have nothing to
do with the present subject and are used as
examples only.
%\end{document}  % This is where a 'short' article might terminate



\appendix
%Appendix A
\section{Headings in Appendices}
The rules about hierarchical headings discussed above for
the body of the article are different in the appendices.
In the \textbf{appendix} environment, the command
\textbf{section} is used to
indicate the start of each Appendix, with alphabetic order
designation (i.e., the first is A, the second B, etc.) and
a title (if you include one).  So, if you need
hierarchical structure
\textit{within} an Appendix, start with \textbf{subsection} as the
highest level. Here is an outline of the body of this
document in Appendix-appropriate form:
\subsection{Introduction}
\subsection{The Body of the Paper}
\subsubsection{Type Changes and  Special Characters}
\subsubsection{Math Equations}
\paragraph{Inline (In-text) Equations}
\paragraph{Display Equations}
\subsubsection{Citations}
\subsubsection{Tables}
\subsubsection{Figures}
\subsubsection{Theorem-like Constructs}
\subsubsection*{A Caveat for the \TeX\ Expert}
\subsection{Conclusions}
\subsection{References}
Generated by bibtex from your \texttt{.bib} file.  Run latex,
then bibtex, then latex twice (to resolve references)
to create the \texttt{.bbl} file.  Insert that \texttt{.bbl}
file into the \texttt{.tex} source file and comment out
the command \texttt{{\char'134}thebibliography}.
% This next section command marks the start of
% Appendix B, and does not continue the present hierarchy
\section{More Help for the Hardy}

Of course, reading the source code is always useful.  The file
\path{acmart.pdf} contains both the user guide and the commented
code.

\begin{acks}
  The authors would like to thank Dr. Yuhua Li for providing the
  matlab code of  the \textit{BEPS} method. 

  The authors would also like to thank the anonymous referees for
  their valuable comments and helpful suggestions. The work is
  supported by the \grantsponsor{GS501100001809}{National Natural
    Science Foundation of
    China}{http://dx.doi.org/10.13039/501100001809} under Grant
  No.:~\grantnum{GS501100001809}{61273304}
  and~\grantnum[http://www.nnsf.cn/youngscientsts]{GS501100001809}{Young
    Scientsts' Support Program}.

\end{acks}

\bibliographystyle{ACM-Reference-Format}
\bibliography{sample-bibliography} 

\end{document}
