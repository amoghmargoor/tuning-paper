\section{Why runtime join reordering does not work}
Adaptive query execution selects alternative execution plans at runtime based on runtime statistics rather than simply executing the compile-time plan. Adaptive query execution works well for selecting the join type, setting the reducers count and input size per reducer but, turns out to be a non-efficient approach for join order planning because:
\begin{itemize}
    \item Blocking execution of the plan
    \item Stats except size are not available
\end{itemize}

To illustrate above claims, let's look at query \texttt{select * from fact\_table f join dim1 d1 on f.k1 = d1.pk join dim2 d2 on f.k2 = d2.pk}. Adaptive query execution would start execution of all the pre-shuffle stages, (henceforth called as a query stage) in parallel. So at the query start, scan of \texttt{fact\_table}, \texttt{dim1}  and \texttt{dim2} tables would start in parallel. As soon as any of this query stage finishes, based on the map output metadata of the all finished query stages, adaptive will try to select a better plan and create new query stages if possible. In the above example, as soon as the query stages corresponding to the scans of \texttt{fact\_table} and \texttt{dim1} finishes, the query stage for the join between those will be started without waiting for the scan of \texttt{dim2} query stage to finish. For adaptive join reordering to work, the query planner will have to wait for all query stages generated at each stage to finish before starting new query stages. In the above example, With blocking execution, join between \texttt{fact\_table} and \texttt{dim1} can only be started after the scan of \texttt{dim1} finishes. Such blocking execution could prove to be counterproductive in the majority of the cases.

Only size statistics are available at each query stage of adaptive execution. Collecting statistics like cardinalities and unique value at each query stage boundary would incur an extra overhead of sampling. This can degrade the performance especially if the join reordering could not come up with a more optimal order.