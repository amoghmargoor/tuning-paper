\subsubsection{Challenges with Adaptive Query for Join Reordering}
Adaptive query execution~\cite{{b12}} selects alternative execution plans at runtime based on runtime statistics rather than simply executing the compile-time plan.
Firstly, this sophisticated technique in not present in any SQL Engine apart from Apache Spark. So any solution built via this will not be applicable to other engines. Secondly, even in Apache Spark it would not help us with multi-way Join Reordering. Adaptive query execution works well for selecting the join type, setting the reducers count and input size per reducer but, turns out to be a non-efficient approach for join order planning because:
\begin{itemize}
    \item It needs blocking the execution of the some part of the plans until all required stages are done.
    \item Stats except table sizes are not available at runtime too.
\end{itemize}

To illustrate above claims, let's look at query:
\begin{verbatim}
    SELECT *
    FROM   fact_table f
    JOIN dim1 d1
    ON f.k1 = d1.pk
    JOIN dim2 d2
    ON f.k2 = d2.pk
\end{verbatim}
Adaptive query execution would start execution of all the pre-shuffle stages, (henceforth called as a query stage) in parallel. So at the query start, scan of \texttt{fact\_table}, \texttt{dim1}  and \texttt{dim2} tables would start in parallel. As soon as any of this query stage finishes, based on the map output metadata of the all finished query stages, adaptive will try to select a better plan and create new query stages if possible. In the above example, as soon as the query stages corresponding to the scans of \texttt{fact\_table} and \texttt{dim1} finishes, the query stage for the join between those will be started without waiting for the scan of \texttt{dim2} query stage to finish. For adaptive join reordering to work, the query planner will have to wait for all query stages generated at each stage to finish before starting new query stages. This is because it needs the runtime statistics of all the sides of all joins in the query to perform Join reordering. In the above example, for join reordering to be done at runtime we would need to block execution untill all 3 tables are scanned. For instance, join between \texttt{fact\_table} and \texttt{dim1} can only be started after the scan of all 3 tables finish. Such blocking execution could prove to be counterproductive in the majority of the cases.

Only size statistics are available at each query stage of adaptive execution. Collecting statistics like cardinalities and unique value at each query stage boundary would incur an extra overhead of sampling. This can degrade the performance especially if the join reordering could not come up with a more optimal order.

However, we believe this can help us at local join reordering i.e., for a single join, it can ensure smaller side is on the right place of the Join to make it efficient. However, it would not be applicable for reordering across multiple joins which is the problem statement for the scope of the paper. 