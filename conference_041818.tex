\documentclass[conference]{IEEEtran}
\IEEEoverridecommandlockouts
% The preceding line is only needed to identify funding in the first footnote. If that is unneeded, please comment it out.
\usepackage{cite}
\usepackage{amsmath,amssymb,amsfonts}
\usepackage{algorithmic}
\usepackage{algorithm}
\usepackage{graphicx}
\usepackage{textcomp}
\usepackage{courier}
\usepackage{xcolor}
\def\BibTeX{{\rm B\kern-.05em{\sc i\kern-.025em b}\kern-.08em
    T\kern-.1667em\lower.7ex\hbox{E}\kern-.125emX}}
\begin{document}

\title{Improving Join Reordering for SQL on Big Data Systems without Statistics\\
}

\author{\IEEEauthorblockN{Amogh Margoor}
\IEEEauthorblockA{\textit{Qubole Inc.} \\
Santa Clara, US}
\and
\IEEEauthorblockN{ Mayur Bhosale}
\IEEEauthorblockA{\textit{Qubole Inc.} \\
Bangalore, India}
}

\maketitle

\begin{abstract}
Using Qubole Data Service, customers process around 4 million SQL Joins per month on their Big Data using Apache Spark. Order in which SQL Joins are performed can vary the efficiency and performance of analytical queries significantly. Hence, finding optimal order for SQL Join is a well researched NP-Hard problem. However, like other traditional Cost Based Optimization in Query processing, even traditional Join Reordering algorithms are not effective on Big Data due to lack of statistics. Statistics are mostly absent in Big Data as they are expensive to compute and needs frequent recompute due to velocity of data change. Hence, ensuring optimal order of Joins is still largely a manual process done via trial and error method on Big Data Systems.

We would like to propose a novel greedy algorithm for Join Reordering which can work in the absence of statistics. Proposed technique is of linear complexity in terms of number of joins. We have observed it improving query performance upto 85\% on TPC-DS benchmarks.
\end{abstract}

\begin{IEEEkeywords}
SQL on Big Data, Query Processing, Join Order, Data Management Systems
\end{IEEEkeywords}

\section{Introduction}
Joins are extremely common for analytical workloads on Big Data and optimizing joins can improve the efficiency and performance of such workloads significantly. Qubole Data Service offers 3 prominent SQL engines for Big Data as a service on public clouds like Amazon AWS, Microsoft Azure and Google Cloud Platform: Apache Hive, Presto and Apache Spark. Customers using Qubole Data Service processes 4 million joins per month using just Apache Spark, making it one of the most commonly used SQL operators. Table \ref{tab:stats} specifies some of the Join metrics we collected for a month on Qubole's offering of Apache Spark. We see around 51\% of joins being INNER JOIN and around 55\% of all Joins using Merge Sort Join algorithm. Due to the frequency and the complexity of Joins, Join Order and Join Strategy are two important decision to be made by Optimizers of Big Data Engines. Optimizers of these Big Data engines like Apache Spark uses traditional Dynamic Programming based algorithms \cite{b1} dependent on column level Statistics. Although Join Reordering is extensively researched problem, the techniques are largely dependent on availability of statistics. For instance, Join reordering algorithm in Spark and Presto will not work without column level statistics. Such extensive statistics are absent for Big Data workloads as they are expensive to compute and maintain. So the Join Reordering rules are never kicked in, leaving it up to the users to take care of it. Since this critical operation is still manual and requires trial and error method, we would like to propose technique which can work on just the table sizes which can be computed during planning stage even if absent in a cheap manner.

In following subsection we would cover in details the importance of Join Reordering for widely used Join algorithms in Presto and Apache Spark. We will then present the current state of Reordering algorithm and Statistics in the prominent Big Data engines and uncover the reasons to consider new algorithm.

\begin{table}[h]
\begin{center}
\begin{tabular}{ |c|c| }
 \hline \\
Joins Processed & 3.9 million \\ \hline \\
Most common Join Type & INNER JOIN - 51\%  \\  \hline \\
Second most common Join Type &  LEFT OUTER JOIN - 31\%\\ \hline  \\
Most common Join algorithm & SortMergeJoin - 55\%\\ \hline \\
Second most common Join algorithm & BroadcastHashJoin - 44\%\\
 \hline
\end{tabular}
\label{tab:stats}
\end{center}
\caption{Join metrics collected over 1 month for Apache Spark in Qubole}
\end{table}

\subsection{Why is Join Reordering important ?}
All the prominent SQL Engines for Big Data like Apache SparkSQL, Presto, Hive etc  have Cost Based Join Reordering which give them multiple Xs in performance improvement\cite{b2}.

Let's look closely into how it can affect a query performance. For illustration, we will pick commonly used Join Algorithm in Presto and Apache Spark. Hash Join is commonly used distributed join in Presto and Sort Merge Join in Apache Spark.

\subsubsection{Distributed Hash Joins in Presto}

\begin{figure}[ht]
\centerline{\includegraphics[width=9.5cm]{fig/DistributedHashJoin.png}}
\caption{Distributed Hash Join}
\label{distributed_hash_join}
\end{figure}

Hash Join is most commonly used distributed join algorithm in Presto.
Figure \ref{distributed_hash_join} illustrates this join.
When joining two tables \texttt{store\_sales} and \texttt{item} Presto regards table on the right i.e., \texttt{item}  as build side and table on left i.e., \texttt{store\_sales} as probe side.
Build side is used to build distributed hash table. As shown in the figure, after table \texttt{item} is scanned, its rows are hash distributed to different nodes based on join key \texttt{i\_item\_sk}. Once distributed, rows from table B are used to create Hash Table. Rows from the Probe Side are also distributed to different nodes using the same hash on the join key \texttt{i\_item\_sk}. Due to this partitioning \texttt{ss\_item\_sk}  and \texttt{i\_item\_sk} with same values end up in the same node.  Rows of Probe Side once distributed are checked against Hash table to find matching join rows. Here, major cost of the join will be building Hash tables on the Build Side. Hence, join reordering can considerably improve the performance by choosing smaller table as build side whenever possible (like in the cases of INNER Joins).

\subsubsection{Sort Merge Join in Spark}

\begin{figure}[ht]
\centerline{\includegraphics[width=9.5cm]{fig/SortMergeJoin.png}}
\caption{Sort Merge Join}
\label{sort_merge_join}
\end{figure}

Spark implements both Sort Merge Join and Hash Join.
Hash Joins are commonly used for Broadcast Joins, where if one of the sides of join is less in size than the threshold set by \texttt{$sql.autoBroadcastJoinThreshold$} then that side's hash table gets broadcasted to nodes with other larger side.
Join is performed locally on those nodes against the broadcasted hash table.

Except for the above case, Sort Merge Join is most commonly used join in Spark. Figure \ref{sort_merge_join} illustrates the Sort Merge Join.
Both tables \texttt{store\_sales} and \texttt{item} are distributed based on hash partitioning on join keys i.e., \texttt{ss\_item\_sk} and \texttt{i\_item\_sk} respectively. Due to this partitioning \texttt{ss\_item\_sk}  and \texttt{i\_item\_sk} with same values end up in the same node. Once rows from each table are distributed, they are individually sorted locally in every node and they are merged together to perform join. Internally Spark while doing merge uses scanner where for every key being considered for merging, one side will be streamed and another side will have rows with merge key buffered. Buffered rows can be spilled to disk too if it crosses memory threshold. In case of INNER Join, left side is streamed and right one will be buffered. So it is better to ensure the buffered side is the smaller one for lesser memory consumption and to avoid spill (spill can affect performance adversely). Hence, Join reordering is essential.

Both the above examples I provided were for a single join. In case of multi-way Join, reordering becomes even more important and can lead to huge performance improvements.
\subsection{Current State of Statistics and Join Reordering Algorithm for Big Data}

Currently, all these engines either use the Hive statistics or have statistics similar to it \cite{b1}:
\begin{itemize}
\item totalSize: Total size of the dataset as its seen at the filesystem level.
\item numFiles: Number of files the dataset consists of.
\item rawDataSize: Uncompressed size of the dataset.
\item numRows: Number of rows in the dataset.
\item columnStatistics: Statistics at column level: number of distinct values, max value, min value, number of null values, max length, average length.
\end{itemize}


In above, subsection we clearly saw why traditional approaches like Cost Based Optimizations and Adaptive Execution not work well for finding optimal Join Order in Big Data. Hence, in this paper we will present a novel greedy approach for Join Reordering which can work without expensive statistics for Big Data Systems.
In the following Section ~\ref{sec:rel_work}, we will discuss the existing work done in academia for Join Reordering and their relevance to our problem statement. Then we will propose our technique and algorithm in Section~\ref{sec:jo}. We will also present experimental evaluation of the algorithm in Section~\ref{sec:exp-evaluation}.

\section{Related Work}
A lot of work has happened in the domain of the query planning of which the join ordering and join strategy are key components. It is agreed upon that the join reordering problem, in particular, is NP-Hard. There are quite a few approximation algorithms that try to solve this problem by leveraging the fact that not all join combinations are valid. System R \cite{b1} first suggested the use of Dynamic Programming (DP) for join ordering and is widely used by the join optimizers. DP based approach like System R come up with all possible valid join combinations and then chooses the plan with minimum cost. Guido and Thomas \cite{b3} propose a graph based dynamic programming strategy by avoiding the combinations which will not lead to valid join trees. The cost function for estimating the cost of running a given plan for both the approach relies heavily on column level statistics like cardinalities, and distinct values for defining the cost function.

The DP based approaches are not scalable as they have exponential run time as well as space complexity. So these techniques are only feasible when the number of joins is small. Spark for example, only reorders joins involving less than 12 relations. Large ad-hoc queries involving more than 10 relations are common in modern-day online analytical processing (OLTP) and object-oriented databases as the SQL queries are often nested. To tackle this problem,  the literature described in [\cite{b4}, \cite{b5}, \cite{b6}] proposes a heuristics-based approach for finding the optimal join order. These systems build the joins greedily either from a bottom-up or a top-down manner. This is different compared to the DP based approaches that rather than minimizing the cost of the entire plan they try to minimize the cost at each step. All though, this may not always produce the most optimal order, the primary goal here is to find a more optimal order compared to the current order.

Both DP and Greedy approached rely on the column level statistic and getting column-level statistics like cardinalities and distinct values require a full table scan.
This makes collecting and maintaining the stats very expensive. [Insert number of cases where stats are not available]. Even if the statistics are available, operators between the scan of the table and join can alter the data making the statistics inaccurate.
Adaptive query execution removes the hard dependency of collecting the up-front statistics and the assumption of their accuracy by reordering the query plan on the fly rather than doing it at the start. In batched query systems like Spark, adaptive collects the statistics after executing all substage below the shuffle boundary.
The newly generated statistics are then used for query replanning on the fly. This procedure is recursively applied at each shuffle boundary to leverage the newly updated statistics. Literature \cite{b7} reorders the nested loop joins after executing every batch of rows from the driver table based on the match rate of the joins. This approach however does not work for other join types because all the tables scans start in parallel. The  RIOS \cite{b8} takes a blocking approach where it starts by executing the pre-shuffle stages and then samples the data thus generated to get the information on cardinality and unique keys. Even though this approach is generic for all join types, the overhead of the sampling the data at shuffle stage and making blocking execution makes this approach infeasible in production environments. [Add numbers]

\section{Join Reordering without statistics}
In prevent section we looked into current state of affair for Join Reordering and why would it not work for Big Data systems. We will like to propose a novel approach in this section to reorder Join just on the basis of table sizes without any elaborate Table or Column Statistics as required by existing approaches.
We have implemented this approach in Apache Spark and will show how our approach gives upto X improvement in TPCDS queries in our empirical evaluation.

\subsection{Heuristics}

Let us look at the assumptions and heuristics used for the approach.

\subsubsection{Assumptions}:

\begin{itemize}
\item In Star Schema Join, the Dimension table’s primary key will be joined with the Fact table’s foreign key.
\item Cardinality of Fact Table joined with Filtered Dimension table will be less than that of the Fact table.
\item So based on last assumption, it follows that if Fact Join with Filtered Dimension happens earlier in multi-way join, it can benefit later joins.
\end{itemize}

To illustrate assumption above, let's look at this query:
\texttt{select * from fact\_table f join dim1 d1 on f.k1 = d1.pk join dim2 d2 on f.k2 = d2.pk where d2.date = ‘06-10-2019’}. If there is a filter on \texttt{dim2} which is smaller than \texttt{dim1} assumption is it’s join with \texttt{fact\_table} would be smaller than \texttt{fact\_table}. Hence, joining \texttt{fact\_table} with \texttt{dim2} and then with \texttt{dim1} will be beneficial.

\subsubsection{Heuristics for detecting Star schema}
In the lack of statistics, we have to use only table sizes and query structure to determine if a Join is a fact table. We are using following heuristics to determine them:

\begin{itemize}
\item Table involved in all of the joins should be a fact table.
\item At least half of the inner joins in the query should involve the fact table.
\item The ratio of the size of the largest dimensions table to fact table size has to be lesser than a certain threshold.
\item None of the dimensions tables is partitioned.
\end{itemize}









\section{Experimental evaluation}\label{sec:exp-evaluation}
We evaluated the above mentioned greedy heuristics-based approach in Apache Spark. Spark's SQL optimizer allows passing the optimizer rules externally to the application. We implemented an optimizer that takes the optimized plan as input and based on the heuristics proposed, reorders the joins wherever applicable. It is assumed that the table sizes are available in the Spark optimized plan before applying this rule. We used the TPC DS~\cite{b14}  dataset (scale 100) non-partitioned dataset for evaluating the performance. The experiments were run on a 5 node Apache Spark cluster on AWS with machine type r4.4xlarge.

Out of 100 TPC-DS~\cite{b14}  queries, at least one join in the 38 queries was reordered. We saw on average  For the queries not reordered, the performance would be the same as before. We saw an average 5\% improvement in perfoormance for the reordered queries. Figure \ref{performance_number} compares the runtimes of the reordered queries with the same query executed with the user order.

\begin{figure}[ht]
    \centerline{\includegraphics[width=7cm]{fig/chart.png}}
    \caption{Performance comparison of reordered queries}
    \label{performance_number}
\end{figure}

Out of all reordered queries, 5.1\% queries showed more than 20\% improvement, 15.4\% queries showed between 10 to 20\% improvement, 51.3\% queries showed between 10 to 20\% improvements and degraded 28.2\% queries showed the degradation of less than 10\%.

\begin{figure}[ht]
\centerline{\includegraphics[width=7cm]{fig/pie.png}}
\caption{Performance distribution of reordered queries}
\label{performance_pie_chart}
\end{figure}

Let's consider query 21 of TPC-DS benchmark. An oversimplified query plan for the query is depicted in Figure \ref{without-reorder}. The join order in the plan is same as one given in the query. \texttt{inventory} table first joins with the \texttt{warehouse} table. The result is then joined with \texttt{item} and \texttt{date\_dim} tables respectively

\begin{figure}[ht]
\centerline{\includegraphics[width=7cm]{fig/without-reorder.png}}
\caption{Query 21 user order}
\label{without-reorder}
\end{figure}

Initially, we extract all consecutive joins from the query plan such that there is only a Project between the two Joins. Once we have the plans of left and right sub-tree and join conditions for each join, we try to extract the dominant Fact table and the dimensions tables. Table \texttt{inventory} is part of the of all three join conditions. So it becomes a candidate for being the fact table and all other tables as dimensions table. In dimensions tables, \texttt{inventory} table has the largest size followed by \texttt{item}, \texttt{date\_dim} and \texttt{warehouse} respectively. The ratio of size of the largest dimensions table, \texttt{item} to the fact table candidate, \texttt{inventory} is 0.008, which is lesser than pre-decided configurable threshold of 0.3. Also, none of the dimension tables is partitioned. So based on these heuristics, we identify that \texttt{inventory} is the dominant fact table and all others are the dimensions table.

After having identified the Fact and Dimensions in the query plan, we check that there are no materialize nodes like UDFs and Explode between the scan and join to ensure that the sizes at the join will be less than or equal to the sizes at the scan. In the example above, there are only \texttt{Project} and \texttt{Filter} nodes between the scans of tables and joins. Also,all the joins are of type \texttt{Inner}. So this example satisfies all the constraints for the Join Reorder and we proceed to reorder the joins if applicable.

Join reorder builds the left deep tree of the joins and constructs them in a bottom-up manner. While building the first join, the plan involving the scan of the identified fact table is selected as the left subtree of the join and right subtree will be chosen from the plans of dimensions table scans. At first, we filter all the plans which have a selective predicate on top of the scan of the dimensions table. If no plan had a selective predicate, we would have taken the plan referred first in the user order. Since in this case, plans corresponding to the scans of \texttt{item} and \texttt{date\_dim} table have a selective predicate, these two plans will be chosen as the potential candidates. Since the table \texttt{date\_dim} has a smaller size out of two, the corresponding plan will be chosen as the right subtree of the first level of the join to be constructed.

In the next steps, a join constructed in the previous iteration, \texttt{inventory Join date\_dim} in our case, will serve as the left subtree for the join to be constructed in this iteration. The right subtree would be picked from the plans of available dimensions table scan similar to the previous iteration. Since only the plan corresponding to the scans of \texttt{item} has a selective predicate, it will be chosen as a right subtree for creating the second level of the join. This procedure is recursively applied until all the dimensions tables are exhausted. So in the third (last) iteration, right subtree would be so far contructed joins and the left subtree would be the plan of only remaining dimensions table, \texttt{warehouse}. Figure \ref{with-reorder} shows the reordered query plan.

\begin{figure}[ht]
\centerline{\includegraphics[width=7cm]{fig/with-reorder.png}}
\caption{Query 21 reordered}
\label{with-reorder}
\end{figure}

\section*{Acknowledgment}

\begin{thebibliography}{00}
\bibitem{b1} Selinger, P. Griffiths, M. M., Astrahan, D. D., Chamberlin, R. A., Lorie, and T. G., Price. ``Access Path Selection in a Relational Database Management System.`` . In Proceedings of the 1979 ACM SIGMOD International Conference on Management of Data (pp. 23–34). Association for Computing Machinery, 1979.
\bibitem{b2} Amogh Margoor, Rajat Venkatesh. ``SQL Join Optimizations in Qubole Presto,``
https://medium.com/qubole-engineering/sql-join-optimizations-in-qubole-presto-3ced3dc75275, unpublished
\bibitem{b3} Guido Moerkotte and Thomas Neumann. 2006. ``Analysis of Two Existing and One New Dynamic Programming Algorithm for the Generation of Optimal Bushy Join Trees without Cross Products.`` In Proceedings of the 32nd International Conference on Very Large Data Bases, Seoul, Korea, September 12-15, 2006. 930–941.
\bibitem{b4} A. Swami ``Optimization of large join queries: combining heuristics and
combinatorial techniques.`` SIGMOD Record, vol. 18, no. 2, 1989.
\bibitem{b5} L. Fegaras ``A new heuristic for optimizing large queries in DEXA ’98``. Proceedings of the 9th International Conference on Database and Expert Systems Applications, 1998.
\bibitem{b6} N. Bruno, C. Galindo-Legaria and M. Joshi ``Polynomial Heuristics for Query Optimization``. 2010 IEEE 26th International Conference on Data Engineering (ICDE 2010), Long Beach, CA, 2010, pp. 589-600, doi: 10.1109/ICDE.2010.5447916.
\bibitem{b7} Li, Quanzhong and Shao, Minglong and Markl, Volker and Beyer, Kevin and Colby, Latha and Lohman, Guy. (2007)  ``Polynomial heuristics for query optimization,`` Proceedings - International Conference on Data Engineering. 26-35. 10.1109/ICDE.2007.367848.
\bibitem{b8} Li, Youfu and Li, Mingda and Ding, Ling and Interlandi, Matteo. (2018)  `` RIOS: Runtime Integrated Optimizer for Spark`` 275-287. 10.1145/3267809.3267814.
\end{thebibliography}

\end{document}
