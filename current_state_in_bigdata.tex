\subsection{Challenges with the Traditional Approaches for Join Reordering in Big Data paradigm}
These two traditional approaches can be used for finding optimal Join order:
\begin{itemize}
\item Cost-Based Optimization
\item Adaptive Execution
\end{itemize}

We will look into the current state of each one of the approaches in Big Data SQL engines and why is it not applicable in most cases.

\subsubsection{Current State of Statistics and Join Reordering Algorithm for Big Data}

Currently, all these engines either use the Hive statistics or have statistics similar to it:
\begin{itemize}
\item totalSize: Total size of the dataset as its seen at the filesystem level.
\item numFiles: Number of files the dataset consists of.
\item rawDataSize: Uncompressed size of the dataset.
\item numRows: Number of rows in the dataset.
\item columnStatistics: Statistics at column level: number of distinct values, max value, min value, number of null values, max length, average length.
\end{itemize}

All of these prominent Big Data SQL engines have Join Reordering algorithms that would need statistics mentioned above. For instance, Apache Spark's Join Reordering is based upon Selinger et al \cite{b1}. It requires the presence of column statistics to work. However, following are the issues with the Statistics in Big Data:

\begin{itemize}
\item Statistics needs to be computed manually by users and hence absent in most workloads.
\item Statistics are expensive to compute, especially on the cloud with per-usage billing. Just to compute column statistics using Apache Spark for 1 table of TPCDS-1000 scale \texttt{store\_sales}, it takes 40 mins job using 5 nodes of r4.2xlarge i.e., an 8 Core machine on AWS. Whereas, our customers process Data at Petabyte scale and would have thousands of such tables, so computing statistics on a regular basis can be an expensive affair especially on cloud due to per-usage billing.
\item Statistics need frequent maintenance. Due to the velocity and volume of Data change in Big Data, statistics computed would go stale pretty frequently and would need frequent updating.
\end{itemize}

Hence, traditional Cost Based Optimization techniques based would not work for Big Data workloads in most of the cases.

However as mentioned in Section~\ref{sec:intro}, Table Sizes are computed on-the-fly by almost all the Engines by listing files under table location and adding their sizes up. This is to ensure when small tables are involved in join, then Broadcast Joins can be performed. This Join optimization can give performance boost. Hence, we can assume table sizes are the only statistics available during query planning for Big Data workload.

\subsubsection{Challenges with Adaptive Query for Join Reordering}
Adaptive query execution~\cite{{b12}} selects alternative execution plans at runtime based on runtime statistics rather than simply executing the compile-time plan.
Firstly, this sophisticated technique in not present in any SQL Engine apart from Apache Spark. So any solution built via this will not be applicable to other engines. Secondly, even in Apache Spark it would not help us with multi-way Join Reordering. Adaptive query execution works well for selecting the join type, setting the reducers count and input size per reducer but, turns out to be a non-efficient approach for join order planning because:
\begin{itemize}
    \item It needs blocking the execution of the some part of the plans until all required stages are done.
    \item Stats except table sizes are not available at runtime too.
\end{itemize}

To illustrate above claims, let's look at query:
\begin{verbatim}
    SELECT *
    FROM   fact_table f
    JOIN dim1 d1
    ON f.k1 = d1.pk
    JOIN dim2 d2
    ON f.k2 = d2.pk
\end{verbatim}
Adaptive query execution would start execution of all the pre-shuffle stages, (henceforth called as a query stage) in parallel. So at the query start, scan of \texttt{fact\_table}, \texttt{dim1}  and \texttt{dim2} tables would start in parallel. As soon as any of this query stage finishes, based on the map output metadata of the all finished query stages, adaptive will try to select a better plan and create new query stages if possible. In the above example, as soon as the query stages corresponding to the scans of \texttt{fact\_table} and \texttt{dim1} finishes, the query stage for the join between those will be started without waiting for the scan of \texttt{dim2} query stage to finish. For adaptive join reordering to work, the query planner will have to wait for all query stages generated at each stage to finish before starting new query stages. This is because it needs the runtime statistics of all the sides of all joins in the query to perform Join reordering. In the above example, for join reordering to be done at runtime we would need to block execution untill all 3 tables are scanned. For instance, join between \texttt{fact\_table} and \texttt{dim1} can only be started after the scan of all 3 tables finish. Such blocking execution could prove to be counterproductive in the majority of the cases.

Only size statistics are available at each query stage of adaptive execution. Collecting statistics like cardinalities and unique value at each query stage boundary would incur an extra overhead of sampling. This can degrade the performance especially if the join reordering could not come up with a more optimal order.

However, we believe this can help us at local join reordering i.e., for a single join, it can ensure smaller side is on the right place of the Join to make it efficient. However, it would not be applicable for reordering across multiple joins which is the problem statement for the scope of the paper. 